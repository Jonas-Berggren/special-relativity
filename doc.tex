\documentclass[12pt]{article}
\usepackage[utf8]{inputenc}
\usepackage[english]{babel}
\usepackage{hyperref}
\hypersetup{
    colorlinks=true,
    linkcolor=black,
    filecolor=black,
    urlcolor=cyan,
}
\urlstyle{same}
\author{{\Huge Jonas Berggren}}
\font\myfont=cmr12 at 30pt
\title{{\myfont Spezielle Relativität}}

\begin{document}
\maketitle
\tableofcontents
\newpage
\begin{abstract}
In diesem Dokument erkläre ich wie ich ein Prgramm entwickelt habe, was Albert Einsteins spezielle Relativität visualisiert.
Außerdem Erkläre ich die darunter liegende Physik und leite Lorentztransformation her.
\end{abstract}
\section{Relativität nach Newtonscher Physik}
Bevor wir über Einsteins spezielle Relativität reden können, müssen wir das Konzept von Raum, Zeit und bewegung klarstellen
\subsection{Bezugssystem}
%position Rotation Zeit, Geschwindgikeit
Zunächst muss klar gestellt werden wie Position, Zeit und Geschwindigkeit gemessen werden.
Dazu muss ein Kordinatensystem Räumlich und Zeitlich definiert werden.
Dass Koordinatensystem hat einen Ursprung  mit $x = 0, y =0, z = 0$ und $t = 0$.
Hierbei ist der Ursprung des Koordinatensystems in der Regel auf ein Objekt zu Beginn des Beobactungszeitraums bezogen.
Für diese Betrachtung müssen folgende Bedingungen erfüllt sein:
\begin{itemize}
\item Das Bezugsystem muss inertial(unbeschleunigt) sein
\item Die Raumzeit muss flach sein, es darf keine Gravitation wirken, ART
\item Alle Inertialen Bezugsysteme sind unbeschleunigt relativ zu einander
\end{itemize}
Durch die Tatsache, dass in allen Inertialen Bezussystemen die gleichen physiklaischen Gesetze gelten, sind alle Bezugssysteme gleich gültig.
Es ist kein Experiment möglich anhand dessen ein Beobachter festellen kann wie schnell er ist oder wo er sich befindet.
Demnach ist es nichts sagend zu sagen, man kätte zum Zeitpunkt $t$ die Position $x, y, z$ und bewge sich mit Gecshwindigkeit $\vec{v}$.
Es muss immer ein Bezugpunkt gewählt werden z.b. Erdmittelpunkt 13. März 2020, 00:00%verbessern
Bezugssysteme können sich also relativ zu eineander bewegen und dennoch gleichermaßen gültig das selbe Ereignis beschreiben.

\subsection{Minkoswsky Raumzeit diargamme}
Das Minkowski Raumzeit Digramm betrachtet, in seiner üblichen Form, Objekte in einer Raumdimension und Zeit.
Hierzu wird die Zeit auf die vertikale Achse gelegt und die Position auf die horizontale.
Für die Betrachtung von spezieller Relativität werden die Einheiten einfachheitshalber so gewählt, das die Lichtgeschwindigkeit $c = 1$ und $x = t$.








x' = x-vt
t' = t
Lichtgeschwindigkeit
x = ct
x' = ct - vt = (c-v)t'
\section{spezielle Relativität}
Jedes Bezugssystem kann als Statisch angenommen werden
speed addition
Minkowskie Raum-Zeit Diagramm
Lichtgeschwindigkeit ist ein Physkalisches Gesetz
Raum, Zeit, Gleichzeitigkeit, Reinfolge
\subsection{Herleitung}
Nehmen wir folgendes Szenarion an:

Es wird ein primärer Beobachter mit dazu gehörigem Bezugssystem betrachtet.
Relativ dazu bewegt sich eine zwiter Beobachter mit dazu gehörigem Bezugssystem.
Die Größen die von letzterem Bezugsystem gemessen werden mit ' bezeichnet.


\subsection{transformation zwischen Bezubssytemen}
Invariante Proper time
propertime, if == 0 they can be connected by a lightbeam
spacelike, timelike
lightcone
Fourvector
Fourvelocity

\subsection{Implikationen}
Gleichzeitigkeit t = vx
Linie der Gleichzeitigkeit ist die Spigelung der Weltline entlang der Lichtlinie
\section{Programm}



\section{Quellen}
Loedel-Minkowski-Diagramm; zweidimensionale Raumzeit(Zugriff: 06.09.2019):

%\url{https://www.youtube.com/watch?v=mJcRHBviHBI}
\url{https://stackoverflow.com/questions/46390231/how-to-create-a-text-input-box-with-pygame}
\url{https://www.youtube.com/playlist?list=PLD9DDFBDC338226CA}
\end{document}
