\documentclass[12pt]{article}
\usepackage[utf8]{inputenc}
\usepackage[english]{babel}
\usepackage{hyperref}
\hypersetup{
    colorlinks=true,
    linkcolor=black,
    filecolor=black,
    urlcolor=cyan,
}
\urlstyle{same}
\author{{\Huge Jonas Berggren}}
\font\myfont=cmr12 at 30pt
\title{{\myfont Spezielle Relativität}}

\begin{document}
\maketitle
\tableofcontents
\newpage
\begin{abstract}
In diesem Dokument erkläre ich wie ich ein Prgramm entwickelt habe, was Albert Einsteins spezielle Relativität visualisiert.
Außerdem Erkläre ich die darunter liegende Physik und leite Lorentztransformation her.
\end{abstract}
\section{Relativität nach Newtonscher Physik}
Bevor wir über Einsteins spezielle Relativität reden können, müssen wir das Konzept von Raum, Zeit und bewegung klarstellen
\subsection{Bezugssystem}
%position Rotation Zeit, Geschwindgikeit
Zunächst muss klar gestellt werden wie Position, Zeit und Geschwindigkeit gemessen werden.
Dazu muss ein Kordinatensystem Räumlich und Zeitlich definiert werden.
Dass Koordinatensystem hat einen Ursprung  mit $x = 0, y =0, z = 0$ und $t = 0$.
Hierbei ist der Ursprung des Koordinatensystems in der Regel auf ein Objekt zu Beginn des Beobactungszeitraums bezogen.
Für diese Betrachtung müssen folgende Bedingungen erfüllt sein:
\begin{itemize}
\item Das Bezugsystem muss inertial(unbeschleunigt) sein
\item Die Raumzeit muss flach sein, es darf keine Gravitation wirken, ART
\end{itemize}
Durch die Tatsache, dass in allen Inertialen Bezussystemen die gleichen physiklaischen Gesetze gelten, sind alle Bezugssysteme gleich gültig.
Es ist keine Universal gülige Aussage über die Position oder Geschwindigkeit eines Körpers, oder Zeitpunkt eines Ereignisses möglich.
Demnach ist es nichts sagend zu sagen, man hätte zum Zeitpunkt $t$ die Position $x, y, z$ und bewge sich mit Gecshwindigkeit $\vec{v}$.
Es muss immer ein Bezugpunkt gewählt werden z.b. Erdmittelpunkt %Ereignis als Zeit reserenz
Bezugssysteme können sich also relativ zu eineander bewegen und dennoch gleichermaßen gültig das selbe Ereignis beschreiben.
\subsection{Wechsel von Bezugssystemen}
Ich werde mich im folgenden auf eine Raumdimension beschränke.
Das Hinzufügen der anderen Raumdimensionen, kann Durch ersetzen der Richtungsabhängigen Größe
n, durch Vekoren.
$x$ wird demnach zu $\vec{0p}$, $v$ zu $\vec{v}$.
dabei ist zu beachten das die gerichteten Relatistischen Effekte nur endlang der Bewgungsrichtung auftreten.

Es wird zunäckst ein bezugssystem gewählt mit den Größen $x, t$ und $v$.
Anschließend wir ein gestrichenes Bezugssystem gewählt mit den Größ $x', t'$ und $v'$, wobei $t = t'$ gilt.
Aus unsere altäglichen erfarung geht hervor, dass für die Position eines, zu dem ungestrichenen Ststem statischen Objekt gilt: $x' = x-vt$
Genau so gilt für geschwindigkeiten:
\begin{equation}
\label{vnewton}
u' =u - v
\end{equation}
Hierbei ist v die Geschwindigkeit des Gestrichenen Bezugsystem und $u$ Die geschwindigkeit des betrachteten objekts.
\subsection{Minkoswsky Raumzeit diargamme}
Das Minkowski Raumzeit Digramm betrachtet, in seiner üblichen Form, Objekte in einer Raumdimension und Zeit.
Hierzu wird die Zeit auf die vertikale Achse gelegt und die Position auf die horizontale.
Für die Betrachtung von spezieller Relativität werden die Einheiten einfachheitshalber so gewählt, dass die Lichtgeschwindigkeit $c = 1$ und $x = t$.
\section{spezielle Relativität}
Die Speziele Relativität fügt ein entscheidendes Postulat hinzu:
\begin{itemize}
\item Die Lichgeschwindigkeit ist ein Universelle Konstante
\end{itemize}
Dies wirft direkt eine Frage auf:
Wenn jemeand mich mit einer Taschenlampe anleuchtet wahrend ich mich auf ihn zu bewge, wie kann es dann sein, dass wir beiden den exakt gleichen wert für die Beschwindigkeit diese licht messen?
Das Pstulat ist also nicht mit der Formel \ref{vnewton} vereinbar.
\subsection{Herleitung}
Nehmen wir folgendes Szenarion an:

Es wir eine Person auf der Erde und eine Person an Bord einer Rakete btrachtete die sich relativ zur Erde mit einer Geschwindikeit $v$ bewegt.
Hierbei ist das Bezugsystem des Astronauten gestrichen.
an Bord der Rakete befindet sich eine Uhr die Zeit misst indem sie ein Photon, über eine Streck $l$ gegen einen Spiegel sendet und warte bis das Photon wiederkommt.
Der Einfachheitskalber bewgt sich dass Licht dabei Orthogonal zur bewegungsrichtung der Rakete.
Aus dem gestrichenen Bezugstystem ist die Zeit die das licht braucht $\Delta t' = \frac{2l}{c}$.
Das licht aus dem Ungestrichenen Bezugsystem jedoch eine längere Streck zurücklegen namlich $s = \sqrt{l^2 + (v \Delta t)^2}$.
Da die Licht geschwindigkeit in Beiden Systemen identisch sein muss, muss $\Delta t' < \Delta t$ gelten.

\subsection{transformation zwischen Bezubssytemen}

\subsection{Implikationen}
\section{Programm}



\section{Quellen}
Loedel-Minkowski-Diagramm; zweidimensionale Raumzeit(Zugriff: 06.09.2019):

%\url{https://www.youtube.com/watch?v=mJcRHBviHBI}
\url{https://stackoverflow.com/questions/46390231/how-to-create-a-text-input-box-with-pygame}
\url{https://www.youtube.com/playlist?list=PLD9DDFBDC338226CA}
\section{Notizen}

x = ct
x' = ct - vt = (c-v)t'
speed addition
Raum, Zeit, Gleichzeitigkeit, Reinfolge
Invariante Proper time
propertime, if == 0 they can be connected by a lightbeam
spacelike, timelike
lightcone
Fourvector
Fourvelocity
Gleichzeitigkeit t = vx
Linie der Gleichzeitigkeit ist die Spigelung der Weltline entlang der Lichtlinie
\end{document}

